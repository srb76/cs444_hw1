\documentclass[letterpaper,10pt,fleqn]{article}
\usepackage{longtable,hyperref}
\newcommand{\longtableendfoot}

\parindent = 0.0 in
\parskip = 0.2 in

\title{Homework 3 - CS444 Fall 2017}
\author{group 45: Jack Neff and Sam Bonner}

\begin{document}
\maketitle
\hrule

\section*{Abstract}

This paper is the writeup submission for Group 45's HW4 assignment. This includes the design plan, version control / work log, and question responses for a modification of slob.c to implement a best-fit allocation algorithm.

\pagebreak
\section*{Design Plan}

In order to implement our best-fit allocation algorithm, we decided to modify the function slob\_page\_alloc. We plan to modify the existing for loop that iterates through pages looking for the first fit, and instead use that for loop to find the best fit page. We will do this by keeping track of the current closest match, and compare subsequent pages against this closest fit. If we find a page that exactly fits, we will interrupt the for loop there, as there will be no closer match. Once we have found our best fit, we will reuse the logic currently present to fragment the head or unlink the page.

\section*{Version Control Log and Work Log}
 
 \begin{tabular}{lp{8cm}}
  \label{tabular:legend:git-log}
  \textbf{acronym} & \textbf{meaning} \\
  V & \texttt{version} \\
  tag & \texttt{git tag} \\
  MF & Number of \texttt{modified files}. \\
  AL & Number of \texttt{added lines}. \\
  DL & Number of \texttt{deleted lines}. \\
\end{tabular}

\bigskip

\begin{longtable}{|rllp{8cm}rrr|}
\hline \multicolumn{1}{|c}{\textbf{V}} & \multicolumn{1}{c}{\textbf{tag}}
& \multicolumn{1}{c}{\textbf{date}}
& \multicolumn{1}{c}{\textbf{commit message}} & \multicolumn{1}{c}{\textbf{MF}}
& \multicolumn{1}{c}{\textbf{AL}} & \multicolumn{1}{c|}{\textbf{DL}} \\ \hline
\endhead

\hline \multicolumn{7}{|r|}{\longtableendfoot} \\ \hline
\endfoot

\hline% \hline
\endlastfoot

\hline 1 &  & 2017-11-29 & Placeholder table & 0 & 0 & 0 \\

\end{longtable}

\section*{Questions}
\begin{enumerate}
\item What do you think the main point of this assignment is?

The main point of this assignment is to have us research and understand how memory is managed and allocated in the SLOB allocator. We also are intended to learn how to implement our own system calls and use them to test our implementations in the kernel.

\item How did you personally approach the problem? Design decisions, algorithm, etc. 

We decided to examine the existing slob.c implementation and modify its page allocation function to search for the best fit in free memory rather than the first possible fit. We did so by iterating through pages in memory, and saving the relevant data (previously implemented in the function) to any location that had less wasted memory than the last best fit. We would iterate until an exact fit was found, or when we reached the last candidate location. If no location could be found with enough memory requested, we would have the function return NULL. 
After finding the data for the best fit, we reused the existing code for allocation after reassigning the best values to those that would have previously been the first fit values.

\item How did you ensure your solution was correct? Testing details, for instance. 

To ensure our solution was correct, we designed a system call to report amount of memory in use.

\item What did you learn? 

We learned how system calls are implemented in the Linux kernel, and how they are invoked within the user space. We also learned how a slab allocator searches for available memory and allocates it to kernel objects.

\item How should the TA evaluate your work? Provide detailed steps to prove correctness.

Ensure that the SLOB allocator is chosen in the menuconfig. In order to do so:
make menuconfig, navigate to General Setup, select the Choose Slab Allocator option, and set it to SLOB.

To patch first fit for comparison, move the ff.patch file outside your linux-yocto-3.19/ directory and run: patch -s -p0 < ff.patch
<How to run test program and interpret output>

To patch best fit for comparison, move the bf.patch file outside your linux-yocto-3.19/ directory and run: patch -s -p0 < bf.patch
Follow the previous steps to run the test program and compare the output to the first fit output.

\end{enumerate}
\end{document}
