\documentclass[letterpaper,10pt,fleqn]{article}

\def\name{Group 45}

\parindent = 0.0 in
\parskip = 0.2 in

\title{Homework 2}
\author{group 45: Jack Neff and Sam Bonner}

\begin{document}
\maketitle
\hrule

\section*{Abstract}

This paper is the writeup submission for Group 45's HW2 assignment. This includes the design plan, version control / work log, and question responses for a C-LOOK implementation of an I/O scheduler.

\pagebreak
\section*{Design Plan}

In order to implement our LOOK I/O scheduler, we planned to use a C-LOOK algorithm. We designed our scheduler to examine requests and order them such that they are placed in an order that operates like the C-LOOK algorithm. This means that we have a integer representing the sector position of the last request dispatched, called head, as a member of the sstf data struct. This head value is compared against the block position of the newest request, such that requests with a sector position after the current head are placed in an ordered fashion in the front segment of the queue, and requests with a block position behind the head are placed at the back segment of the queue. This means requests will be dispatched in one direction, then the head will return to the next earliest sector position, and do another sweep of all requests positioned after that.

\section{Version Control Log and Work Log}
 

\section*{Questions}
\begin{enumerate}
\item What do you think the main point of this assignment is?

This assignment was intended to guide us into investigating and learning about how I/O schedulers, the Linux block layer, and elevator algorithms operate. This assignment is also intended to better familiarize us with qemu command options.

\item How did you personally approach the problem? Design decisions, algorithm, etc. 

We first examined the NO-OP I/O scheduler and researched how LOOK and C-LOOK algorithms operated. We chose to implement a C-LOOK algorithm as it seemed to be simpler to understand and optimize. After learning how Linux list heads were implemented, and after examining how the elevator.c functions operated, we decided to implement our C-LOOK scheduler by examining the sector position of each request currently in the queue and comparing it against the sector position of the request to be added to the queue. We implemented logic to place requests ahead of the current head's sector position in a sorted order in front of the current head, and requests the are behind the current head's sector position after those requests in sorted order. This caused requests to be dispatched while travelling in one direction and then jumping back to the earliest sector position to travel in that direction again. This logic and implementation modelled a C-LOOK approach to I/O scheduling.

\item How did you ensure your solution was correct? Testing details, for instance. 

We made use of printk statements inside the dispatch and add request functions to report what sector positions they were reading or writing to. We examined this output to prove that our requests were being dispatched in a C-LOOK model.

\item What did you learn? 

We learned about how the block layer, elevator algorithms, and I/O schedulers operate to organize and carry out read and write requests on a drive. We also learned how to disable the virtual I/O used on our previous qemu command line.

\item How should the TA evaluate your work? Provide detailed steps to prove correctness.



\end{enumerate}

\begin{thebibliography}{9}

\end{thebibliography}





\end{document}
