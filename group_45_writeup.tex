\documentclass[letterpaper,10pt,fleqn]{article}

\def\name{Group 45}

\parindent = 0.0 in
\parskip = 0.2 in

\title{Homework 1}
\author{group 45: Jack Neff and Sam Bonner}

\begin{document}
\maketitle
\hrule

\section*{Booting the Kernel in Qemu}

Command log:
From directory fal2017/45/yocto-linux-3.19.2

Terminal 1: 
make mrproper

cp ../config-3.10.2-yocto-standard .config

make menuconfig

make -j4 all

source ../environment-setup-i586-poky-linux

qemu-system-i386 -gdb: tcp::5545 -S -nographic -kernel arch/i386/boot/bzImage -drive file=../core-image-lsb-sdk-qemux86.ext4,if=virtio -enable-kvm -net none -usb -localtime --no-reboot --append "root=/dev/vda rw console=ttyS0 debug"

Terminal 2: 
source ../environment-setup-i586-poky-linux

\$GDB

target remote localhost:5545

continue

Terminal 1: 

(when prompted) root

uname -r

We confirmed we were running the right kernel with the following method:

1. Upon calling make menuconfig with terminal 1, changed the Local Version from yocto-standard to group45kernel. 
2. After booting the kernel with qemu, using the command uname -r to print the local version. Upon seeing it was "group45kernel," we knew it was correct. 

\section{Qemu Command Line}
qemu-system-i386 -gdb: tcp::5545 -S -nographic -kernel bzImage-qemux86.bin -drive file=core-image-lsb-sdk-qemux86.ext4,if=virtio -enable-kvm -net none -usb -localtime --no-reboot --append "root=/dev/vda rw console=ttyS0 debug"


Flags: 

-gdb: This flag is followed by a device (in this case localhost:5545) and signals to wait for a gdb connection on that device. 

-S: Do not start CPU at startup. This is why the command halts until you type "continue".

-nographic: Disables the graphics and turns qemu into a command line. 

-kernel: Signals to use the following file as the kernel image. This one uses bzImage-qemux86.bin.

-drive: Defines a new drive. The drive is used with the disk image at the file indicated by the option file=core-image-lsb-sdk-qemux86.ext4, and is connected to the virtio interfce by the if= option. 

-enable-kvm: Enables full Kernel-based Virtual Machine support, turning the linux kernel create and run virtual machines.

-net none: No network devices are being configured here. 

-usb: Enables usb driver. We can upload files to the VM via usb. 

-localtime: Boots the kernel with the current UTC.

--no-reboot: Instead of allowing the user to reboot, just exit the program. 

--append: Passes command line arguments to the kernel boot. These are located in /dev/hda. ttys0 is the Linux port used, and the kernel is started in debug mode.
 

\section*{Concurrency Questions}
\begin{enumerate}
\item What do you think the main point of this assignment is?

This assignment was intended to introduce or refresh our knowledge of parallel programming, by requiring
us to implement a multithreaded program with shared resources.

\item How did you personally approach the problem? Design decisions, algorithm, etc. 

We first researched the producer-consumer model and the operation of mutex locks. The example provided by the textbook Modern Operating Systems [1] was a useful starting point and teaching material for this exercise. As the rdrand instruction was not supported on the os2 servers, we chose to include the Mersenne Twister C implementation [2] in our code. We utilized Mersenne Twister by creating a function that would generate a number within the given range passed to the function (two ints). This was done by usng the Mersenne Twister function genrand real2 to acquire a double between 0 and 1, then multiplying the maximum value by this double until a value above the minimum was obtained. The shared buffer we implemented was an array of 32 Item structs. The buffer operated simply by placing an item in the first available space, denoted by an Item.number value of 0. Items were consumed from the buffer in the same manner, by finding the first available non-zero Item, saving the Item.number and Item.wait values in a new Item, then resetting that Item's members back to zero, and returning the saved Item. Consumer sleep time was read from the returning struct, and producer sleep time was generated as a random value.

\item How did you ensure your solution was correct? Testing details, for instance. 

Several printf statements were used at key points in program operation, including when a buffer location was accessed, when the producer created an item, and when the consumer removed an item. The program was also left to run until 100 items had been produced and consumed, and output was checked to determine if the expected output was produced.

\item What did you learn? 

We learned how mutex locks and pthread waits actually operate and how to implemented them correctly. This included learning how pthread cond signals are used to unblock threads, and when to call them.

\end{enumerate}

\begin{thebibliography}{9}
\bibitem{Tanenbaum} 
Tanenbaum, Andrew S.
\textit{Modern Operating Systems}. 
Pearson, 2007.
\\\texttt{http://cis.poly.edu/cs3224a/Code/ProducerConsumerUsingPthreads.c}
\bibitem{mt}
Matsumoto, Makoto and Nishimura, Takuji.
\textit{Mersenne Twister}
2002.
\\\texttt{http://www.math.sci.hiroshima-u.ac.jp/~m-mat/MT/MT2002/CODES/mt19937ar.c}
\end{thebibliography}

\end{document}
