\documentclass[letterpaper,10pt,fleqn]{article}
\usepackage{longtable,hyperref}
\newcommand{\longtableendfoot}

\parindent = 0.0 in
\parskip = 0.2 in

\title{Homework 3 - CS444 Fall 2017}
\author{group 45: Jack Neff and Sam Bonner}

\begin{document}
\maketitle
\hrule

\section*{Abstract}

This paper is the writeup submission for Group 45's HW3 assignment. This includes the design plan, version control / work log, and question responses for a RAM disk device driver that encrypts writes and decrypts reads.

\pagebreak
\section*{Design Plan}

In order to implement our encrypted block device, we will utilize the sbd driver available here: \newline
\url{http://blog.superpat.com/2010/05/04/a-simple-block-driver-for-linux-kernel-2-6-31/}
We will use AES encryption from crypto.h to encrypt our reads and writes in the sbd transfer function. In order to use encryption, we will first need to initialize the AES cipher within the program's init function, and set the key obtained from a module parameter. To encrypt reads and writes, we will attempt to encrypt byte by byte, by utilizing the crypto cipher encrypt one function in crypto.h.

\section*{Version Control Log and Work Log}
 
 \begin{tabular}{lp{8cm}}
  \label{tabular:legend:git-log}
  \textbf{acronym} & \textbf{meaning} \\
  V & \texttt{version} \\
  tag & \texttt{git tag} \\
  MF & Number of \texttt{modified files}. \\
  AL & Number of \texttt{added lines}. \\
  DL & Number of \texttt{deleted lines}. \\
\end{tabular}

\bigskip

\begin{longtable}{|rllp{8cm}rrr|}
\hline \multicolumn{1}{|c}{\textbf{V}} & \multicolumn{1}{c}{\textbf{tag}}
& \multicolumn{1}{c}{\textbf{date}}
& \multicolumn{1}{c}{\textbf{commit message}} & \multicolumn{1}{c}{\textbf{MF}}
& \multicolumn{1}{c}{\textbf{AL}} & \multicolumn{1}{c|}{\textbf{DL}} \\ \hline
\endhead

\hline \multicolumn{7}{|r|}{\longtableendfoot} \\ \hline
\endfoot

\hline% \hline
\endlastfoot

\hline 1 &  & 2017-11-12 & Added module c files to Github & 2 & 366 & 0 \\
\hline 2 &  & 2017-11-12 & Added initialization of cipher and setting key & 1 & 28 & 3 \\
\hline 3 &  & 2017-11-12 & Added encryption / decryption byte by byte & 1 & 13 & 4 \\
\hline 4 &  & 2017-11-12 & Changed decryption function to save into buffer & 1 & 1 & 1 \\
\hline 5 &  & 2017-11-13 & Implemented changes from Patterson's blog & 1 & 5 & 4 \\
\hline 6 &  & 2017-11-13 & Changed key module parameter & 1 & 4 & 4 \\
\hline 7 &  & 2017-11-14 & Added first draft of writeup & 1 & 93 & 0 \\
\hline 8 &  & 2017-11-14 & Added proj3, basically an sbd.c copy for debugging & 2 & 240 & 0 \\
\hline 9 &  & 2017-11-14 & Added proj3, basically an sbd.c copy for debugging & 2 & 0 & 16 \\
\hline 10 &  & 2017-11-14 & Rough draft writeup & 1 & 4 & 34 \\
\hline 11 &  & 2017-11-15 & finalized module code & 1 & 186 & 160 \\
\hline 12 &  & 2017-11-15 & Added info to design plan & 1 & 1 & 0 \\
\hline 13 &  & 2017-11-15 & Added title section ebd.c & 1 & 5 & 2 \\

\end{longtable}

\section*{Questions}
\begin{enumerate}
\item What do you think the main point of this assignment is?

The main point of this assignment is to teach us about block devices, modules, and encryption in th Linux kernel, and also how to work with a piece of software that there isn't much community support for. At times it felt like the point of this assignment was to teach us how to comb the internet for the right piece of code to base our solution on. 

\item How did you personally approach the problem? Design decisions, algorithm, etc. 

We decided the first order of business would be figuring out how to get a module (or any file for that matter, something we had yet to accomplish) onto the vm from the os2 server. Once we had this methodology down we were able to freely build new modules on os2 and then transfer them into a running vm where we could test them. For design, we found superpat's blog with a simple block driver for 2.6 Kernel, and then way down in the comments we found this guy Sarge who had updates a few lines to port the driver to the 3.2 kernel. Bingo, all that was left was encryption. For the encryption, we did some pretty standard print statement debugging and internet searching to understand the syntax of the crypto functions. From there it just took some analysis of the sbd request function to realize where to place them.


\item How did you ensure your solution was correct? Testing details, for instance. 

To ensure correctness, we examined both encrypted and unencrypted data from our driver. Once data had been encrypted, it appeared totally different.


\item What did you learn? 

We learned how to configure a block device driver into a kernel, how to scp between a server and a vm, and got a glimpse under the hood at how block I/O works. From reading through usr/log/messages for planted printk statements, we were able to examine encrypted and unencrypted data being written to the block device. 

\item How should the TA evaluate your work? Provide detailed steps to prove correctness.

Start in the vm. SCP the module into the root directory. Then do the following:
1. insmod 10000
2. 

\end{enumerate}
\end{document}
