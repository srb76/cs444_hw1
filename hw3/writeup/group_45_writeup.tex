\documentclass[letterpaper,10pt,fleqn]{article}
\usepackage{longtable,hyperref}
\newcommand{\longtableendfoot}

\parindent = 0.0 in
\parskip = 0.2 in

\title{Homework 3 - CS444 Fall 2017}
\author{group 45: Jack Neff and Sam Bonner}

\begin{document}
\maketitle
\hrule

\section*{Abstract}

This paper is the writeup submission for Group 45's HW3 assignment. This includes the design plan, version control / work log, and question responses for a RAM disk device driver that encrypts writes and decrypts reads.

\pagebreak
\section*{Design Plan}

In order to implement our encrypted block device, we will utilize the sbd driver available here: \url{http://blog.superpat.com/2010/05/04/a-simple-block-driver-for-linux-kernel-2-6-31/}

\section*{Version Control Log and Work Log}
 
 \begin{tabular}{lp{8cm}}
  \label{tabular:legend:git-log}
  \textbf{acronym} & \textbf{meaning} \\
  V & \texttt{version} \\
  tag & \texttt{git tag} \\
  MF & Number of \texttt{modified files}. \\
  AL & Number of \texttt{added lines}. \\
  DL & Number of \texttt{deleted lines}. \\
\end{tabular}

\bigskip

\section*{Questions}
\begin{enumerate}
\item What do you think the main point of this assignment is?

The main point of this assignment is to teach us about block devices, modules, and encryption in th Linux kernel, and also how to work with a piece of software that there isn't much community support for. At times it felt like the point of this assignment was to teach us how to comb the internet for the right piece of code to base our solution on. 

\item How did you personally approach the problem? Design decisions, algorithm, etc. 

We decided the first order of business would be figuring out how to get a module (or any file for that matter, something we had yet to accomplish) onto the vm from the os2 server. Once we had this methodology down we were able to freely build new modules on os2 and then transfer them into a running vm where we could test them. For design, we found superpat's blog with a simple block driver for 2.6 Kernel, and then way down in the comments we found this guy Sarge who had updates a few lines to port the driver to the 3.2 kernel. Bingo, all that was left was encryption. For the encryption, we did some pretty standard print statement debugging and internet searching to understand the syntax of the crypto functions. From there it just took some analysis of the sbd_request function to realize where to place them.


\item How did you ensure your solution was correct? Testing details, for instance. 

To ensure correctness, we examined both encrypted and unencrypted data from our driver. Once data had been encrypted, it appeared totally different.


\item What did you learn? 

We learned how to configure a block device driver into a kernel, how to scp between a server and a vm, and got a glimpse under the hood at how block I/O works. From reading through usr/log/messages for planted printk statements, we were able to examine encrypted and unencrypted data being written to the block device. 

\item How should the TA evaluate your work? Provide detailed steps to prove correctness.



\end{enumerate}
\end{document}
